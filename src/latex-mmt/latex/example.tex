\documentclass{article}

%\usepackage[utf8]{inputenc}
%\DeclareUnicodeCharacter{27F6}{\longrightarrow}
%\DeclareUnicodeCharacter{2218}{\circ}

\usepackage{url}
\usepackage{xcolor}
\usepackage{listings}
% we use listings only for source code examples; it needs some help handling unicode characters
\lstdefinelanguage{unicode}{
  literate={⟶}{{$\longrightarrow$}}2{∘}{{$\circ$}}1
}
\lstset{mathescape,basicstyle=\footnotesize\color{gray},language=unicode}


\usepackage{amsmath}

%\usepackage{fdsymbol}
\newcommand{\rightassert}{\vdash} % should be supplied by \usepackage{fdsymbol}, but that causes a weird error

% change this to give your MathHub root folder
\newcommand{\mmttexmathhubroot}{c:/other/oaff/}
\usepackage{mmttex}

%\RequireMMTPackage{MMT/examples/export/presentation/mmt-sty/content/http..cds.omdoc.org/examples/$P$L}

\begin{document}

\title{MMTTeX: Integrating Formal Mathematics in MMT and Narrative Mathematics in LaTeX}
\author{Florian Rabe}
\date{FAU Erlangen and LRI Paris}
\maketitle

\begin{abstract}
This document serves as an announcement, documentation, and example of the MMTTeX tool that allows integrating formal/logical content written in MMT with informal/narrative one written in LaTeX.
Background knowledge is formalized in MMT and can be exported as LaTeX sty files.
Then documents written in LaTeX and may import this background knowledge.
The documents may declare further formal knowledge, which is interpreted by MMT on the fly during the LaTeX compilation workflow.

In particular, documents may contain mathematical formulas in MMT syntax: MMT parses and type-checks these formulas, enriches them by inferring types and implicit arguments, and eventually produces LaTeX presentations for them.
The latter then contains annotations generated by MMT such as tooltips and cross-references.

This document is itself written in MMTTeX.
\end{abstract}

\section{Formalizing the Background Knowledge}

The MMT system\footnote{\url{https://uniformal.github.io/}} provides a wide variety of logical frameworks and logics that are readily available for use with \lstinline|mmttex|.
Alternatively, users may formalize novel background knowledge in MMT.

For this document, we assume the definition of first-order logic in the \lstinline|MMT/examples| archive.
It can be found at \url{https://gl.mathhub.info/MMT/examples/tree/master/source/logic}.

\section{Writing LaTeX Documents}

MMTTeX documents generate two kinds of output: the usual presentation output produced by LaTeX, e.g., a pdf file, and a content output in the form of an MMT file.
Correspondingly, we have to distinguish two kinds of MMTTeX commands:
\begin{itemize}
  \item \textbf{Content-only} commands declare formal knowledge that is independent of the presentation output.
  They set up the context in which MMT processes the presentation-relevant commands.
  They occur in the content output and produce no or almost no presentation output.
  \item \textbf{Presentation-relevant} commands produce output in the usual LaTeX way.
\end{itemize}

This document is itself an example of how to use MMTTeX, and all MMTTeX commands are duplicated as listings.
As a running example, we build a theory of groups.

\subsection{Namespace Abbreviations}

These are content-only commands that declare namespace abbreviations to be used later.

%%%%%%%%%%%%%%%%%%
\begin{lstlisting}
\mmtimport{ex}{http://cds.omdoc.org/examples}
\end{lstlisting}
\mmtimport{ex}{http://cds.omdoc.org/examples}
%%%

\subsection{MMT Theories}

The environment \lstinline|mmttheory| is used to group the content-only commands into scopes.
The begin/end commands of this environment are content-only, but the environment may contain arbitrary presentation-relevant commands.
It takes two arguments: the name of the theory and the MMT URI of its meta-theory.

%%%%%%%%%%%%%%%%%%
\begin{lstlisting}
\begin{mmttheory}{Group}{ex:?FOLEQNatDed}
\end{lstlisting}
\begin{mmttheory}{Group}{ex:?FOLEQNatDed}
%%%

Now we have opened a theory named \lstinline|Group| whose meta-theory is (using the namespace abbreviation from above) \lstinline|http://cds.omdoc.org/examples?FOLEQNatDed|, a formalization of first-order logic from MMT's \lstinline|examples| archive.

\subsection{Constant Declarations}

We declare the basic operations of monoids:

%%%%%%%%%%%%%%%%%%
\begin{lstlisting}
\mmtconstant{op}{term ⟶ term ⟶ term}{}{1 ∘ 2 prec 5}
\mmtconstant{unit}{term}{}{e}
\mmtconstant{inv}{term ⟶ term}{}{1 ' prec 10}
\end{lstlisting}
\mmtconstant{op}{term ⟶ term ⟶ term}{}{1 ∘ 2}
\mmtconstant{unit}{term}{}{e}
\mmtconstant{inv}{term ⟶ term}{}{1 '}
%%%

Here the content-only command \lstinline|mmtconstant| takes $4$ arguments: name, type, definiens, and notation.
Apart from the names, all may be omitted by leaving them empty.
\lstinline|prec| sets the precedence, i.e., inversion binds stronger than composition.

As examples of defined constants, we define the commutator and division:

%%%%%%%%%%%%%%%%%%
\begin{lstlisting}
\mmtconstant{division}{term ⟶ term ⟶ term}{[x,y] x∘y'}{1 / 2}
\mmtconstant{commutator}{term ⟶ term ⟶ term}{[x,y] x∘y∘x'∘y'}{< 1 , 2 >}
\end{lstlisting}
\mmtconstant{division}{term ⟶ term ⟶ term}{[x,y] x∘y'}{1 / 2}
\mmtconstant{commutator}{term ⟶ term ⟶ term}{[x,y] x∘y∘x'∘y'}{< 1 , 2 >}
%%%

\subsection{MMT-Processed Formulas}

So far all commands were content-only.
Now we say the main example of a presentation-relevant command.
The syntax \lstinline|"..."| can be used as an analog to LaTeX's usual \lstinline|$...$|.
But the formula is interpreted by MMT.

As example, we use the commutator function again:
The function "[x,y] x∘y∘x'∘y'" takes two group elements and returns their commutator.
Using the above definition, we can also write it as "[x,y]<x,y>".

Often we need to open local scopes in which we declare variables.
For that, we use the content-only \lstinline|mmtcontext| environment.
Contexts can be nested.

Inside a context, we can use the \lstinline|mmtvar| command to declare variables.
It combines content and presentation aspects: it produces a content declaration of a variable that can be referred to in the same scope and the presentation of the variable name.
There is also a content-only variant \lstinline|mmtvar*|.

\begin{mmtcontext}
First we declare a variable as part of the text: consider a group element \mmtvar{a}{term}.
Now we can use it to write objects like "a'".
\begin{mmtcontext}
We silently declare a second variable, which we use below.\mmtvar*{b}{term}

Now we can say that "a∘b∘a'∘b'" is the commutator of "a" and "b".
\end{mmtcontext}
\end{mmtcontext}

\subsection{Closing Theories}

Eventually, we close our theory of groups:

%%%%%%%%%%%%%%%%%%
\begin{lstlisting}
\end{mmttheory}
\end{lstlisting}
\end{mmttheory}
%%%

\subsection{Theory Inclusions}

Theories may include other theories.
It does not matter whether a theory is part of the background knowledge or declared in the current document.

For example, we create a new theory that includes the one above and the natural numbers from the background knowledge:

%%%%%%%%%%%%%%%%%%
\begin{lstlisting}
\begin{mmttheory}{Group2}{ex:?FOLEQNatDed}
\mmtinclude{?Group}
\mmtinclude{ex:?Nat}
\end{lstlisting}
\begin{mmttheory}{Group2}{ex:?FOLEQNatDed}
\mmtinclude{?Group}
\mmtinclude{ex:?Nat}
%%%

In this theory, we declare a power operation:

%%%%%%%%%%%%%%%%%%
\begin{lstlisting}
\mmtconstant{power}{term ⟶ nat ⟶ term}{}{1 ^ 2}
\end{lstlisting}
\mmtconstant{power}{term ⟶ nat ⟶ term}{}{1 ^ 2}
%%%

Finally, we close the theory:

%%%%%%%%%%%%%%%%%%
\begin{lstlisting}
\end{mmttheory}
\end{lstlisting}
\end{mmttheory}
%%%


\section{Semantic Enrichment}

\begin{mmttheory}{Group3}{ex:?FOLEQNatDed}
\mmtinclude{?Group2}

Consider the MMT-processed formula \lstinline|"[n,x,y]x'^n / y"|, which is rendered as "[n,x,y] x'^n / y".
\end{mmttheory}

The rendered formula contains a number of advanced features:
\begin{itemize}
\item Several delimiters are interpreted as two-dimensional notations including \lstinline|^| for superscripting and \lstinline|/| for fractions.
\item Every operator carries a hyperlink: try clicking on the $\lambda$ or the inversion operator.
\item The inferred types of the bound variables occur as tooltips over the variables: try hovering over the binding of $x$.
\end{itemize}

\section{Compilation Workflow}\label{sec:workflow}

MMTTeX is activated by including the \lstinline|mmttex.sty| package.

In line with typical LaTeX workflows, MMTTeX requires two compilations of the source file with MMTTeX run in between.

\subsection{Background Knowledge}

MMT provides the build target \lstinline|mmt-sty|.
It is used to converts the background knowledge into LaTeX packages to make it available to LaTeX documents.
It is executed as
\lstinline|build ARCHIVE mmt-sty|, where \lstinline|ARCHIVE| is the name of an MMT archive.
It produces a LaTeX \lstinline|sty| file for each MMT theory.

For example, to compile this document, users should run
\begin{lstlisting}
build MMT/urtheories mmt-sty
build MMT/examples mmt-sty
\end{lstlisting}
in the MMT shell.

\subsection{MMT Server}

\lstinline|mmttex| can be run as a self-contained tool.
However, if a LaTeX document uses a lot of background knowledge, it is advisable to run MMT permanently as a server so that the background does not have to be reloaded every time.
This has the additional advantage that hyperlinks to the background knowledge can be resolved in the web browser.

To start an MMT server, you can execute for example
\begin{lstlisting}
mmt --keepalive server on 8080
\end{lstlisting}


\subsection{First LaTeX Pass}

The first LaTeX pass exports all MMT content into an \lstinline|JOB.tex.mmt| file that is produced in addition to the usual output files.

\paragraph{Example}
For this document itself, the file is (except for possibly some Unicode characters that are not displayed correctly when including it into LaTeX):

% this file is currently open for writing and normally closed at the end of the document; we need to force-close it so that we can list it
\immediate\closeout\mmtfile
\lstinputlisting[breaklines,frame=topline]{example.tex.mmt}

\paragraph{Explanations}
Two things are particularly interesting:
\begin{itemize}
\item All \lstinline|mmtcontext| environments become nested theories with auto-generated names.
\item All presentation-relevant objects become defined constants, whose names are of the form \lstinline|mmt@| followed by section number and a formula relative to the section.
\end{itemize}

\subsection{MMTTeX}

Next MMTTeX reads this file and produces the file \lstinline|JOB.tex.mmt.sty|, where \lstinline|JOB| is the main LaTeX file.
For this document, assuming all paths are set up correctly, this is simply the command
\begin{lstlisting}
mmttex example.tex
\end{lstlisting}
Any type-checking errors are printed to standard output.

\paragraph{Example}
For this document, the resulting file is:
\lstinputlisting[mathescape=off,breaklines,frame=topline]{example.tex.mmt.sty}

\paragraph{Explanations}
With some effort, one can see the following:
\begin{itemize}
\item All theories from the background result in \lstinline|RequireMMTPackage| commands.
\item Every constants results in a macro definition, whose name is derived from the name of the containing theory and the constants.
 Types and definitions are ignored, and the notation of the constant is used to define the macro.
\item Every MMT-processed formula results in a macro definition using the constant name.
 The definiens of the macro is the type-checked form of the object rendered using a set of macros that are defined by the \lstinline|mmttex| package.
\end{itemize}

These special macros include
\begin{itemize}
 \item Every delimiter is wrapped in \lstinline|mmt@symref| and carries the MMT URI of the MMT constant that constructed the term.
 The macro definition uses this to create a cross-reference to the place where the constant was declared.
 \item \lstinline|mmt@group| indicates the exact structure of the syntax tree, and \lstinline|mmt@vardecl| is used for variable declarations.
 \item \lstinline|mmt@vardecl| is used for variable declarations. If the type of the variable was inferred, it is additionally passed in MMT surface syntax, and the \lstinline|mmt@vardecl| macro uses this to add the inferred as a tooltip on the variable.
\end{itemize}

\subsection{Second LaTeX Pass}

In the second LaTeX pass, \lstinline|mmttex.sty| detects the presence of the generated style file.
Now every time it encounters an MMT-processed formula, the macro produced in the previous step is executed, thus inserting the appropriate LaTeX object.

%\subsection{Old Part}
%
%\begin{mmttheory}{test}{ex:?FOLEQNatDed}
%
%\mmtconstant{id}{prop ⟶ prop}{[x] x}{id 1}
%
%"[A] impI [p: ded A] andI p p"
%
%\begin{mmtcontext}
%\mmtvar{x}{prop}
%
%\mmtvar{y}{prop}
%
%\begin{mmtcontext}
%\mmtvar{z}{prop}
%\mmtvar{zz}{prop}
%"x ⇒ y ⇒ z ⇒ zz"
%\end{mmtcontext}
%
%\mmtvar{yy}{prop}
%"[a,b] a ⇒ b ⇒ yy"
%
%\end{mmtcontext}
%"id true"
%
%\end{mmttheory}

\end{document}