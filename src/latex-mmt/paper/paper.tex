\documentclass[orivec]{llncs}
\pagestyle{plain}

\usepackage{url,xcolor,paralist}
\usepackage{wrapfig}
\usepackage{amsmath,amssymb}
\usepackage{xspace}

\usepackage{listings}

\usepackage{graphicx}
%\usepackage{xparse} % custom argument syntax

\newcommand{\mmttex}{MMTTeX\xspace}
\newcommand{\sTeX}{sTeX\xspace}

\setlength{\hfuzz}{3pt} \hbadness=10001
\setcounter{tocdepth}{2} % for pdf bookmarks

\usepackage[bookmarks,linkcolor=red,citecolor=blue,urlcolor=gray,colorlinks,breaklinks,bookmarksopen,bookmarksnumbered]{hyperref}

%%%%%%%%%%%%%%%%%%%%%%%%%%%%%%%%%%%%%%%%%%%%%%%%%%%%%%%%%%%%%%%%%%%%%%%
% local macros and configurations
\usepackage{basics}

\usepackage{local}

\newcommand{\mmttexmathhubroot}{c:/other/oaff/}
\usepackage{../latex/mmttex}

\mmtimport{ex}{http://cds.omdoc.org/examples}

\title{MMTTeX: Connecting Content and Narration-Oriented Document Formats}
\author{Florian Rabe}
\institute{Universities Paris-Sud and Erlangen-Nuremberg}

%%%%%%%%%%%%%%%%%%%%%%%%%%%%%%%%%%%%%%%%%%%%%%%%%%%%%%%%%%%%%%%%%%%%%%%
\begin{document}

\maketitle
\begin{abstract}
Narrative, presentation-oriented assistant systems for mathematics such as \latex on the one hand and formal, content-oriented ones such as proof assistants and computer algebra systems on the other hand have so far been developed and used largely independently.
The former excel at communicating mathematical knowledge and the latter at certifying its correctness.

\mmttex aims at combining the advantages of the two paradigms.
Concretely, we use \latex for the narrative and {\mmt} for the content-oriented representation.
Formal objects may be written in MMT and imported into \latex documents or written in the \latex document directly.
In the latter case, \mmt parses and checks the formal content during \latex compilation and substitutes it with \latex presentation macros.

Besides checking the formal objects, this allows generating higher-quality \latex than could easily be produced by hand, e.g., by inserting hyperlinks and tooltips into formulas.
Moreover, it allows reusing formalizations across narrative documents as well as between formal and narrative ones.
As a case study, the present document was already written with \mmttex.
\end{abstract}


\lstMakeShortInline[basicstyle=]!
\lstset{mathescape,basicstyle={\footnotesize\color{gray}}}

\section{Introduction}\label{sec:intro}
  %A compelling vision for upcoming decades in mathematical content creation is to build a system that automatically verifies mathematical documents as they are written.
%State of the art systems tend to focus on either informal, present-oriented document creation or content-oriented formalization, with {\LaTeX} resp. proof assistants being the leading systems.

A major open problem in mathematical document authoring is to elegantly combine formal and informal mathematical knowledge.
Multiple proof assistants and controlled natural language systems have developed custom formats for that purpose, e.g., \cite{isabelle_documentoriented,mizar,plato,naproche}.
%Mizar's document format \cite{mizar} has always been designed to closely resemble common narrative mathematical language.
%Recent work for Isabelle opens up the Isabelle kernel in order to permit better integration with other applications, in particular document-oriented editors\cite{isabelle_documentoriented}.
%%\cite{largeformalwikis} develops a general Wiki infrastructure that can use Mizar and Coq as verification backends.
%In the Plato system \cite{plato}, TeXMacs is used as a document-oriented frontend whose documents are parsed and verified by the Omega proof assistant.
%Also, several proof assistants can export a {\LaTeX} version of their native input language (e.g., Isabelle and Mizar).
%Similarly, most can export documents and libraries as HTML, which provides added-value features like navigation and dynamic display.
%The exported documents do not fully utilize the narrative flexibility of {\LaTeX} or HTML and are not meant for further editing.
%Controlled natural language systems such as Naproche \cite{CramerFKKSV09}, MathLang \cite{KMW:MAEDLoM}, and MathNat \cite{HumayounMNMTCNL} support much more flexible informal language than proof assistants, and some permit \latex fragments as a part of their input language.

%A major bottleneck has been that both proof assistants and controlled natural language systems are usually based on a fixed foundation (a logic in which all formalizations are carried out) and a fixed implementation that verifies documents written in a native input language.
%Therefore, the formalization of a mathematical document $D$ in a proof assistant usually requires the -- expensive -- formalization of the background theory that connects $D$ to the foundation.
%Moreover, the fixed implementation is limited to its input language, which in turn cannot be fully understood by any other system.

The goal of \mmttex is not to introduce a new format but to use \latex documents as the primary user-written documents.
That makes it most similar to \sTeX \cite{stex}, where authors add content markup that allows converting a \latex to an OMDoc document.
%Via \latex ML \cite{latexml}, this permits generating \omdoc from the same source code that produces PDF documents.
%\omdoc itself \cite{omdoc} is an XML format that integrates both content and narration markup.
%Both \omdoc and sTeX do not define a reference semantics that would permit verifying the documents.
%However, they are also sufficiently generic to eventually allow interfacing with verification systems in principle.
\mmttex differs in several ways: authors write \emph{fully formal} content in \mmt syntax \cite{RK:mmt:10} directly in the \latex source, either in addition to or instead of informal text; and \mmt is used to type-check the formal parts during \latex compilation workflow.
This enables several advanced features:
Formal content in \latex sources can use or be used by \mmt content written elsewhere; in particular, background knowledge formalized elsewhere can be used inside the \latex document.
And formulas written in \mmt syntax are not only type-checked but result in high-quality pdf by, e.g., displaying inferred types as tooltips or adding hyperlinks to operator occurrences.

%{\mmt} \cite{RK:mmt:10} has been developed as a foundation-independent document format coupled with an application-independent implementation.
%Foundation-independence means that {\mmt} makes no assumptions about the underlying logic and instead represents this logic explicitly as an {\mmt} theory itself.
%This provides the flexibility to adapt the language to any formal system used to describe a mathematical domain.
%Application-independence means that the {\mmt} system focuses on building a simple and transparent API along with reusable services.
%This permits integrating {\mmt} easily into concrete applications and workflows.

%In this paper, we leverage and evaluate {\mmt} by integrating its content-oriented services with the \latex document format.
%We extend \mmt with a concrete input language for mathematical formulas and use that to supplement \latex's own formula language.
%\mmt processes these formulas and transforms them into plain \latex.
%This has three advantages:
%\begin{inparaenum}[\it i\rm)]
%\item formulas become easy to write and read (e.g., \lstinline|1+2 in Z|) while still providing the benefits of content markup (like in \lstinline|\isin\plus{x}{y}\Z|),
%\item \mmt can apply type reconstruction services to infer implicit types and arguments and to signal errors for ill-typed formulas,
%\item \mmt can produce enriched formulas that include hyper-references, tooltips, and interactive display.
%\end{inparaenum}
%
%We overcome two major challenges to integrate this processing with existing \latex workflows.
%Firstly, on the technical side, the formulas have to be recognized in the {\latex} document, processed by {\mmt}, and replaced with generated {\latex} fragments that are inserted into the document.
%Secondly, on the theoretical side, we have to account for the fact that formulas are usually not self-contained and depend on a context defined both by the containing document and other documents, e.g., cited \latex documents or formalizations of background theories in proof assistants.
%Therefore, we have to make these dependencies explicit in the \latex source and make \mmt aware of them.

%We introduce the input language for \mmt in Sect.~\ref{sec:not-lang} and solve the above theoretical challenge in Sect.~\ref{sec:doc-format}.
%Then we provide two solutions to the technical challenge by integrating \mmt with the pdflatex and the \latexml \cite{latexml} processors in Sect.~\ref{sec:pdflatex} and~\ref{sec:latexml}, respectively.
%
%We exemplify our development using a running example comprising \mmt theories for the logical framework LF \cite{lf}, sorted-first order logic (SFOL) defined in LF, and monoids and monoid actions defined in SFOL.
%The present paper is already written in \mmt-\latex.
%Therefore, it contains the running example not only as \latex listings but also as processed \latex, which permits experiencing the added-value functionality directly.

\paragraph{Online Resources}
All resources are available as a part of the \mmt repository, specifically at \url{https://github.com/UniFormal/MMT/tree/devel/src/latex-mmt} for the current version.
These resources include the \mmt and \latex side of the implementation, the system documentation, and the sources of this paper, which is itself written with \mmttex.

\paragraph{Acknowledgments}
Both this paper and the \mmttex implementation were redone from scratch by the author.
But that work benefited substantially from two unpublished prototypes that were previously developed in collaboration with Deyan Ginev, Mihnea Iancu, and Michael Kohlhase.
The author was supported by DFG grant RA-18723-1 OAF and EU grant Horizon 2020 ERI 676541 OpenDreamKit.

%While users will be willing to sacrifice some automation support for that, \mmt must still provide substantial automation support.
%Every formalization of mathematical knowledge is carried out in the most natural and weakest foundation, which is itself defined in the framework, and formalizations can be reused and moved between foundations.
%All assistance functionality is realized in reusable libraries and services that make no assumptions about the application they are used in.
%This permits integrating assistant systems with each other and with other systems such as web browsers, IDEs, and domain-specific document processors.
%The combination of application- and foundation-independence maximizes interoperability and reuse.

%The IMPS system \cite{imps} is based on a variant of higher-order logic with partial functions.
%Regarding mathematical knowledge, Wiedijk identifies \cite{qedrevisited} HOL Light, Coq, ProofPower, Mizar, and Isabelle/HOL as the most advanced systems using a sample of 100 representative mathematical theorems.
%In addition, many systems support complex conservative extension principles, such as type definitions in HOL, provably terminating recursive functions in Coq or Isabelle/HOL\ednote{check}, or provably well-defined indirect definitions in Mizar.

%Foundation-independence goes beyond the similar approach of logical frameworks \cite{logicalframeworks} such as Twelf \cite{twelf} based on the dependent type theory $\cn{LF}$ \cite{lf} and \cite{isabelle} based on intuitionistic higher-order logic.
%{\pn} represents even these as particular foundations, whose semantics is defined within the {\pn} framework using the narrative or computational paradigm.
%Reuse extends to these definitions as well: In the example on the right, $\cn{LF}^<$, an extension of $\cn{LF}$ with subtyping, is used to define PVS and ZFC set theory.

%Dually, executable papers\footnote{Elsevier Executable Paper Grand Challenge, \url{http://www.executablepapers.com/}} or computable documents\footnote{Wolfram, Computable Document Format, \url{http://www.wolfram.com/cdf/}} supplement narration with computation.

% Combine narrative and type-checking worlds; MMT and LaTeX(ML) need to communicate. Achieve formal content type-checking and verification inside LaTeX.

\section{Design and Behavior} \label{sec:pdflatex}
  \subsection{Overview}

MMTTeX consists of two components:
\begin{compactitem}
 \item an \mmt plugin \texttt{latex-mmt} that converts \mmt theories to \latex packages,
 \item a small \latex package \texttt{mmttex.sty} (about $100$ loc with only standard dependencies), which allows for embedding \mmt content.
\end{compactitem}
The two components are tied together in bibtex-style, i.e.,
\begin{compactenum}
 \item While running \latex on \texttt{doc.tex}, \texttt{mmttex.sty} produces an additional output file \texttt{doc.tex.mmt}, which is a regular \mmt file.
 \texttt{doc.tex.mmt} contains all formal content that was embedded in the \texttt{tex} document.
 \item \texttt{latex-mmt} is run on \texttt{doc.tex.mmt} and produces \texttt{doc.tex.mmt.sty}.
  This type-checks the embedded formal content and generates macro definitions for rendering it in the following step.
 \item When running \latex the next time, the package \texttt{doc.tex.mmt.sty} is included at the beginning. Now all embedded formal content is rendered using the macro definitions from the previous step. If the formal content changed, \texttt{doc.tex.mmt} also changes.
\end{compactenum}

\noindent
Note that \texttt{latex-mmt} only needs to be re-run if the formal content document changed.
That is important for sharing documents with colleagues or publishers who want to or can only run plain \latex: by simply supplying \texttt{doc.tex.mmt.sty} along with \texttt{doc.tex}, running plain \latex is sufficient to build \texttt{doc.pdf}.

\subsection{Formal Content in LaTeX Documents}

% Throughout this document, !text! is an abbreviation for \verb|text|

\texttt{mmttex.sty} provides presentation-\emph{irrelevant} and presentation-\emph{relevant} macros for embedding formal content in addition to resp. instead of informal text.
\medskip

\textbf{Presentation-irrelevant macros} only affect \texttt{doc.tex.mmt} and do not produce anything that is visible in the pdf document.
These macros can be used to embed a (partial) formalization of the informal text.
The formalization can occur as a single big chunk, be interspersed with the \latex source akin to parallel markup, or be anything in between.
Importantly, if split into multiple chunks, one formal chunk may introduce names that are referred to in other formal chunks, and \latex environments are used to build nested scopes for these names.

At the lowest level, this is implemented by a single macro that takes a string and appends it to \texttt{doc.tex.mmt}.
On top, we provide a suite of syntactic sugar that mimics the structure of the \mmt language.

As a simple example, we will now define the theory of groups informally and embed its parallel \mmt formalization into this paper.
Of course, the embedded formalization is invisible in the pdf.
Therefore, we added \lstinline|listings in gray| that show the presentation-irrelevant macros that occur in the \latex sources of this paper and that embed the formalization.
If this is confusing, readers may want to inspect the source code of this paper at the URL given above.

Our example will refer to the theory !SFOLEQ!, which formalizes sorted first-order logic with equality and is defined in the \texttt{examples} archive of \mmt.\footnote{See \url{https://gl.mathhub.info/MMT/examples/blob/master/source/logic/sfol.mmt}.}
To refer to it conveniently, we will import its namespace under the abbreviation !ex!.

\begin{mmttheory}{Group}{ex:?SFOLEQ}
\begin{lstlisting}
\mmtimport{ex}{http://cds.omdoc.org/examples}
\begin{mmttheory}{Group}{ex:?SFOLEQ}
\end{lstlisting}

A group consists of
\begin{compactitem}
 \item a set $U$,
\mmtconstant{U}{sort}{}{}
\begin{lstlisting}
\mmtconstant{U}{sort}{}{}
\end{lstlisting}

 \item an operation $U\to U \to U$, written as infix $*$,
\mmtconstant{operation}{tm U --> tm U --> tm U}{}{1 * 2 prec 50}
\begin{lstlisting}
\mmtconstant{operation}{tm U --> tm U --> tm U}{}{1 * 2 prec 50}
\end{lstlisting}

 \item an element $e$ of $U$ called the unit
\mmtconstant{unit}{tm U}{}{e}
\begin{lstlisting}
\mmtconstant{unit}{tm U}{}{e}
\end{lstlisting}

\item an inverse element function $U\to U$, written as postfix $'$ and with higher precedence than $*$.
\mmtconstant{inv}{tm U --> tm U}{}{1 ' prec 60}
\begin{lstlisting}
\mmtconstant{inv}{tm U --> tm U}{}{1 ' prec 60}
\end{lstlisting}
\end{compactitem}
We omit the axioms.

\end{mmttheory}
\begin{lstlisting}
\end{mmttheory}
\end{lstlisting}

Here the environment !mmttheory! wraps around the theory.
It takes two arguments: the name and the meta-theory, i.e., the logic in which the theory is written.

The macro !mmtconstant! introduces a constant declaration inside a theory.
It takes $4$ arguments: the name, type, definiens, and notation. All but the name may be empty.

We can also use the \mmt module system, e.g., the following creates a theory that extends !Group! with a definition of division (where square brackets are the notation for $\lambda$-abstraction employed by !SFOLEQ!):

\begin{lstlisting}
\begin{mmttheory}{Division}{ex:?SFOLEQ}
\mmtinclude{?Group}
\mmtconstant{division}
   {tm U --> tm U --> tm U}{[x,y] x*y'}{1 / 2 prec 50}
\end{lstlisting}

\begin{mmttheory}{Division}{ex:?SFOLEQ}
\mmtinclude{?Group}
\mmtconstant{division}{tm U --> tm U --> tm U}{[x,y] x*y'}{1 / 2 prec 50}

\noindent
Note that we have not closed the theory yet, i.e., future formal objects will be processed in the scope of !Division!.
\medskip

\textbf{Presentation-relevant macros} result in changes to the pdf document.
The most important such macro provided by \texttt{mmttex.sty} is one that takes a math formula in \mmt syntax and parses, type-checks, and renders it.
For this macro, we provide special syntax that allows using quotes instead of dollars to have formulas processed by \mmt:
if we write !"$F$"! (including the quotes) instead of $\mathdollar F\mathdollar$, then $F$ is considered to be a math formula in \mmt syntax and processed by \mmt.
For example, the formula !"forall [x] x / x = e"! is parsed by \mmt relative to the current theory, i.e., !Division!; then \mmt type-checks it and substitutes it with \latex commands. In the previous sentence, the \latex source of the quoted formula is additionally wrapped into a verbatim macro to avoid processing by \mmt; if we remove that wrapper, the quoted formula is rendered into pdf as "forall [x] x / x = e".

Type checking the above formula infers the type !tm U! of the bound variable !x!.
This is shown as a tooltip when hovering over the binding location of !x!.
(Tooltip display is supported by many but not all pdf viewers.
Unfortunately, pdf tooltips  are limited to strings so that we cannot show tooltips containing \latex or MathML renderings even though we could generate them easily.)
Similarly, the sort argument of equality is inferred.
Moreover, every operator carries a hyperlink to the point of its declaration.
%This permits easy type and definition lookup by clicking on symbols.
Currently, these links point to the \mmt server, which is assumed to run locally.
%An alternative solution is to put the inferred type into the margin or next to the formula.
%Then we can make it visible upon click using pdf JavaScript.
%As this is poorly supported by pdf viewers though, we propose another solution: Similar to a list of figures, we produce a list of formulas, in which every numbered formula occurs with all inferred parts.
%This has the additional benefit that the added value is preserved even in the printed version.
\medskip

\noindent
This is implemented as follows:
\begin{compactenum}
 \item An \mmt formula !"$F$"! simply produces a macro call !\mmt@X! for a running counter !X!.
 If that macro is undefined, a placeholder is shown and the user is warned that a second compilation is needed.
 Additionally, a definition !mmt@X = $F$! in \mmt syntax is placed into \texttt{doc.tex.mmt}.
 \item When \texttt{latex-mmt} processes that definition, it additionally generates a macro definition !\newcommand{\mmt@X}{$\ov{F}$}! in \texttt{doc.tex.mmt.sty}, where $\ov{F}$ is the intended \latex representation of $F$.
 \item During the next compilation, \mmt@X produces the rendering of $\ov{F}$.
 If $F$ did not type-check, additional a \latex error is produced with the error message.
\end{compactenum}

\noindent
Before we continue, we close the current theory:
\end{mmttheory}
\begin{lstlisting}
\end{mmttheory}
\end{lstlisting}


%Actually, the above example is simplified -- the snippets returned by \mmt are much more complex in order to produce semantically enriched output.
%For the pdf workflow, we have realized three example features.
%Recall that the present paper is an \mmt-\latex document so that these features can be tried out directly.


\subsection{Converting MMT Content To LaTeX}

%In a second step, the background is made available to \latex in the form of \latex packages.
%\mmt already provides a build management infrastructure that allows writing exporters to other formats.

We run \texttt{latex-mmt} on every theory $T$ that is part of the background knowledge, e.g., !SFOLEQ!, and on all theories that are part of \texttt{doc.tex.mmt}, resulting in one \latex package (sty file) each.
This package contains one \lstinline|\RequirePackage| macro for every dependency and one \lstinline|\newcommand| macro for every constant declaration.
\texttt{doc.tex.mmt.sty} is automatically included at the beginning of the document and thus brings all necessary generated \latex packages into scope.

The generated \lstinline|\newcommand| macros use (only) the notation of the constant.
For example, for the constant named !operator! from above, the generated command is essentially !\newcommand{\operator}[2]{#1*#2}!.
Technically, however, the macro definition is much more complex:
Firstly, instead of !#1*#2!, we produce a a macro definition that generates the right tooltips, hyperreferences, etc.
Secondly, instead of !\operator!, we use the fully qualified \mmt URI as the \latex macro name to avoid ambiguity when multiple theories define constants of the same local name.

The latter is an important technical difference between \mmttex and \sTeX \cite{stex}: \sTeX intentionally generates short human-friendly macro names because they are meant to be called by humans.
That requires relatively complex scope management and dynamic loading of macro definitions to avoid ambiguity.
But that is inessential in our case because our macros are called by generated \latex commands (namely those in the definiens of !\mmt@X!).
Nonetheless, it would be easy to add macros to \texttt{mmttex.sty} for creating aliases with human-friendly names.

The conversion from \mmt to \latex can be easily run in batch mode so that any content available in \mmt can be easily used as background knowledge in \latex documents.

%sTeX uses a more complex mechanism that defines macros only when a theory is included and undefines them when the theory goes out of scope.
%This does not solve the name clash problem and is more brittle when flexibly switching scopes in a document.
%But it results in human-usable macro names, which is the main design goal of sTeX.

%In particular, this allows using \lstinline|\mmt@symref{$U$}{\wedge}| instead of \lstinline|\wedge|, where $U$ is as above.
%\lstinline|\mmt@symref| is defined in \lstinline|mmttex| and adds a hyperref and a tooltip to the $\wedge$ symbol.
%We will see this in action below.

  
\section{Conclusion}\label{sec:conc}
  We have presented a system that allows embedding formal \mmt content inside \latex documents.
The formal content is type-checked in a way that does not affect any existing \latex work flows and results in higher quality \latex than could be easily  produced manually.
Moreover, the formal content may use and be used by any \mmt content formalized elsewhere, which allows interlinking across document formats.

Of course, we are not able to verify the informal parts of a document this way --- only those formulas that are written in \mmt syntax are checked.
But our design supports both gradual formalization and parallel formal-informal representations.
%Firstly, it can act as a bridge technology that exposes a wider audience of authors of mathematical texts to formalization tools while retaining their existing workflows.
%Secondly, it enables exporting the high-level structure of a document and its remaining proof obligations for further refinement in a proof assistant.
%This helps synchronize and establish cross-references between the mathematical document and its detailed formalization.

It is intriguing to apply the same technology to formal proofs.
This is already possible for formal proof terms, but those often bear little resemblance to informal proofs.
Once \mmt supports a language for structured proofs, that could be used to write formal proofs in \mmttex.
Morever, future work could apply \mmt as a middleware between \latex and other tools, e.g., \mmt could run a computation through a computer algebra system to verify a computation in \latex.


\bibliographystyle{alpha}
\bibliography{../../../../Program_Data/Latex/bib/rabe,../../../../Program_Data/Latex/bib/systems,../../../../Program_Data/Latex/bib//pub_rabe}

\end{document}
